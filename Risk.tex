
\subsection*{Risk and Liability{{\color{red}\ *}}}
The prototype for this project and its implicated future applications carry inherent risk due to the human interaction with moving parts and machinery. While the ultimate goal of this research is to enhance the interaction between humans and machines, one or both of them may fail during operation.

Specifically for the lab prototype there are certain fail-safes implemented to prevent injury. While not as large scale as some potential industrial and commercial applications, there is still risk to the user. The track, with the motor moving the belt and cart at potentially high velocities presents a pinching or collision hazard. Although the user's hand shouldn't be near the belt during operation, should such an event occur there is a switch box with an emergency stop button connected to the amplifier that can be pressed and will cut all power to the motor immediately. Through the switch box there also runs a set of limit switches that sit at the ends of the track. These prevent the cart from running off the end of the track or into the motor. If the user inputs a force that exceeds the limits of the system, a beeping noise will be triggered through the myRIO as a warning to lower or completely remove the input. Once the force input is back in the allowable range the beeping stops and the program is ready to continue.

The future industrial and commercial uses for this type of impedance control have more risk than the prototype. Implementation in a warehouse or factory will undoubtedly introduce more axes of motion. The use of more axes requires the user to be more aware of their surroundings as items are moved around. Care must also be taken when inputting parameters to ensure correct values are being used. Incorrect values may cause the system to not work and cause damage to the machine or human. If the system fails while lifting a heavy mass, the user will be unable to move or stop the mass, potentially endangering the user, bystanders, or the environment.
