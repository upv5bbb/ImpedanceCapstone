\subsubsection*{ --- Continued Development}
There are improvements and new ideas that build upon the existing prototype to better the research into variable impedance controllers. With the current setup, only one cell is used, which reduces cost, but also decreases the directional sensitivity of the system. There is a slight difference in voltage range when the force input is into the load cell versus when the force is in the direction of the M10 bolt. Using two identical load cells would ensure that the voltage range in either direction is the same. The current load cell also has a force detection range of up to 50 pounds. This is much higher than the needs of the prototype setup. Generally with lower force detection ranges, load cells become more accurate. With a smaller-range load cell, the force measurement readings would become more accurate. Using a metal cart instead of a PLA cart like the one in use would increase the overall stiffness of the system. Having a metal cart would eliminate the need for the rod and plate configuration being used to reduce the effects of creep in the PLA. Before being applied to industrial practices, multiple axis integration must be proven. Moving to a multiple axis prototype would allow for testing of scenarios that can't be examined with the current one-dimensional setup.