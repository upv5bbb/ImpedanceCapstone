\subsubsection*{ --- Software}
Much of the software for the system was a modification from the previous quarter's embedded computing labs. The C-Code basically implements a closed-loop control system for a DC motor with a virtual reference model. The system controls the motor speed by comparing the actual velocity of the motor with the desired reference velocity. This is done through three threads: the main program thread, timer thread, and the table update thread.

The main thread accomplishes all the initialization and configuration such as the ctable structure, IRQ channel, load cell, and encoder interface. Two threads are created using \verb|pthread_create()| which start the routines \verb|Timer_Irq_Thread()| and \verb|Table_Update_Thread()|. After the initializations, it runs the table editor by calling \verb|ctable()| which runs continuously until the backspace button is entered in the keypad. When \verb|ctable()| ends, the two threads are terminated, timer IRQ is unregistered, and the MyRIO session ends.

The \verb|Timer_Irq_Thread()| schedules the next interrupt using \verb|NiFpga_WriteU32()| to write the interval time into register using the timeout value of 5000 microseconds. The timeout value is defined by the BTI value from the ctable structure. \verb|NiFpga_WriteBool()| is called to set the time register flag to true. When the interrupt is signaled, the service routine does a few things. First it calibrates the load cell by reading the analog input of the load cell voltage 10,000 times. Then the average is taken to use for the initial load cell voltage. Force reading is calculated by taking the difference of the input load cell voltage and the initial load cell voltage, then multiplying by 124.1897 which is determined from experimental data. A deadband is implemented to reject any voltage reading under 1 Newton. This will insure the cart stays stationary when there is no input force and help eliminate noise to the load cell.

Next the reference system is calculated using the biquad cascade and Tustin's approximation explained in the Computational section. If the control voltage reaches the limit voltage, it will set \verb|VLimitFlag| which will cut off the force input into the reference system. This will slow down the reference system will slow down and the physical cart will be able to catch up. \verb|cascade()| is used to calculate the reference velocity which will be used to compare to the actual velocity. \verb|vel()| is used to measure the actual velocity of the motor which is calculated by reading the encoder counts. \verb|cascade()| calculates the control voltage using the PI controller and error between the actual and reference velocity. There is an if-statement that checks for output voltage limitation and will only output the maximum and minimum values as well as warning noises to ensure safety. Then \verb|Aio_Write()| is used to send the control voltage to the DAC. After each iteration, the interrupt service is acknowledged with \verb|Irq_Acknowledge|. The thread will terminate when signaled.

The \verb|Table_Update_Thread()| has no interrupt and is only terminated by the ready flag. It periodically calls \verb|update()| which refreshes the LCD screen display values. The timing of the update is based on \verb|nanosleep()| which waits for imprecisely 0.5 seconds. 