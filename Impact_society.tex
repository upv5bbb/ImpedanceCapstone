%---Impact on Society
\subsection*{Impact on Society}
%Help Aging population get back to work, safe working conditions, rehabilitation
Impedance control systems are utilized in many different forms all around the world to improve people's lives, from helping senior citizens getting back into the workforce to physical rehabilitation.

In Japan and Singapore, where the aging population are straining the country's workforce, people have been looking into technologies which can help the senior citizens go back to work again. To do that, they came up with solutions which implemented impedance controls method such as motor-assisted pallet trucks and motor-assisted suit so as to help the seniors push and carry heavier loads and be on par with their younger counterparts. Moreover, these solutions can also be used by their younger counterparts to reduce strain on their body and create a safer working condition for everyone.

For many who had suffered injury to their arm, stroke, or spinal cord, they need to undergo rehabilitation treatment in order to regain full function and control of their arm again. By using a robot with a position-based impedance controller, the therapist will be able to ask the robot to assist or resist movement of the user's wrist, elbow, or shoulder in order to simulate arm movement conditions in daily life.

%Creates competition


%Bibliography/References