\subsubsection*{ --- Description \& Implementation}
The prototype requires a computer with a C compiler and the code included in the appendix. Before running, the MyRIO must be plugged into the computer and connected. The 25 V power supply connected to the load cell must be turned on and set to 5 V. The oscilloscope should be turned on and auto-scaled. The 24 V motor amplifier should be turned on. The red LED on the solderless bread board should be off, if not the green button on the reset switch will turn it off, indicating that the circuit is complete. If pressing the green button does not cause the LED to turn off, the cart is likely engaging one of the limit switches and should be moved to the center of the track. 
\newline \indent Now the prototype can be used and the code can be run. There is a base set of virtual parameters in the code that can be seen on the LCD screen attached to the MyRIO by scrolling through the display. If the user wishes to edit one of the virtual parameters, he or she must scroll until the desired parameter to be changed is on top, or they may hit the number of that parameter to bring it automatically to the top. Once it is at the top, the user can press enter on the keypad and then type the desired value (keeping in mind the functional specifications mapping) and hit enter. The new virtual parameter will be set and the prototype's response to force inputs will change accordingly. 
\newline \indent To apply a force and observe the behavior, the user grabs the cart by the handle and pushes either direction. The user may also choose to simply prod the handle in one direction or another to provide an impulse force. 
\newline \indent When testing is complete, the user hits the backspace arrow on the keypad to exit the program. If, during testing the cart reaches the end of the track and engages a limit switch, the amplifier will no longer provide current to the motor. It is very important to not reset the switch using the green button until after the program is exited. Doing otherwise could result in damage to the prototype. If, during testing, the keypad emits a short chirp sound, it is because the force being input exceeds the functional specifications limitations for the given virtual system. When testing is complete and the program is exited, data for the force and velocity both of the real and virtual systems are exported to a MATLAB file for analysis.