\subsubsection*{ --- Hardware}
A lot of equipment was used to complete the design. A National Instruments MyRIO was the microcomputer used, which runs Linux and features a Field Programmable Gate Array (FPGA). The MyRIO has 4 connectors, designated A through D. On one of these connections is an LCD screen and keypad that is used for the user input and some data tracking while the program runs. On another of the MyRIO connections is a Quanser Terminal Board, a device that allows direct connection of two 5 pin DIN and 4 BNC type cables. The DIN inputs are for encoders and the BNC connectors are for Digital to Analog and Analog to Digital converters (DAC and ADC, respectively). The DAC is connected to a 24 V current amplifier with a maximum current of 4.8 A. The output of the amplifier goes directly to a Pittman 9236 brushed DC motor with an E30 series 512 pulse optical encoder and 5.9:1 gearbox. The encoder is connected to the 0th encoder input on the Quanser board. The motor is mounted to an Open Builds Part Store 1.5 m V-slot belt-driven linear actuator. Mounted onto the ends of the track are two Open Build Part Store micro limit switches. The limit switches are wired through a reset switch and two mechanical relays and back into the amplifier. On the cart is a TE Connectivity Measurement Specialties FC22 Compression Load Cell. The load cell is powered by an HP signal generator at 5 V. The output of the load cell runs to the 0th ADC connector on the Quanser board as well as an oscilloscope. 