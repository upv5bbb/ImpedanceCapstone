\subsubsection*{C-Code}
\item{Control Code}
\begin{verbatim}
/* Impedance Control */

/* includes */
#include "stdio.h"
#include "MyRio.h"
#include "me477.h"
#include <string.h>
#include "ctable.h"
#include "TimerIRQ.h"
#include "IRQConfigure.h"
#include <pthread.h>
#include "Encoder.h"
#include "math.h"
#include "matlabfiles.h"

struct biquad {
double b0; double b1; double b2; 
// numerator
double a0; double a1; double a2; 
// denominator
double x0; double x1; double x2; 
// input
double y1; double y2; }; // output
//int timeoutValue;
#include "ControllerHeader.h"

/* prototypes */
void *Timer_Irq_Thread(void* resource);
void *Table_Update_Thread(void* resource);
double cascade(int ns, double xin, 
struct biquad *fa);
double vel(void);
void wait(void);

/* definitions */
MyRio_Aio CI0;
MyRio_Aio CO0;
typedef struct {NiFpga_IrqContext irqContext; 
// context
struct table *a_table; // table
NiFpga_Bool irqThreadRdy; // ready flag
} ThreadResource;
typedef struct {int looptime_ns; // table
NiFpga_Bool upThreadRdy; // ready flag
} upThreadResource;
NiFpga_Session myrio_session ;
static MyRio_Encoder encC0;
#define IMAX 5000 // max points
#define VLimit 5 // max VDAout
static double IniForce, Total=0.0;
static int k = 1,l=1; // force benchmarking

int main(int argc, char **argv)
{
NiFpga_Status status;

status = MyRio_Open();  	
/*Open the myRIO NiFpga Session.*/
if (MyRio_IsNotSuccess(status)) 
return status;

// Initialize Table
char *Table_Title ="Control Table";
static struct table my_table[] = {
{"V_ref: m/s ", 0, 0.0 },
{"V_act: m/s ", 0, 0.0 },
{"VDAout: mV ", 0, 0.0 },
{"M: kg ", 1, 5.0 },
{"K: N/m ", 1, 5.0 },
{"B: N*s/m ", 1, 25.0 },
{"BTI: ms ", 1, 5.0 },
{"Force: N", 0, 0.0}
};

MyRio_IrqTimer irqTimer0;
ThreadResource irqThread0;
upThreadResource updateThread;
pthread_t thread;
pthread_t thread2;
// Specify IRQ channel settings
irqTimer0.timerWrite = IRQTIMERWRITE;
irqTimer0.timerSet = IRQTIMERSETTIME;
timeoutValue = 5000;
// Initialize analog interfaces 
// before allowing IRQ
AIO_initialize(&CI0, &CO0); 
// initialize analog I/O
Aio_Write(&CO0, 0.0); 
// zero analog output
// Initialize encoder
EncoderC_initialize(myrio_session,&encC0);
// Configure Timer IRQ. 
//Terminate if not successful
status = Irq_RegisterTimerIrq(
&irqTimer0,
&irqThread0.irqContext,
timeoutValue);
// Set the indicator 
// to allow the new thread.
irqThread0.irqThreadRdy = NiFpga_True;
irqThread0.a_table = my_table;
// Create new thread to catch the IRQ.
status = pthread_create(&thread,
NULL,
Timer_Irq_Thread,
&irqThread0);
// Create new thread to update table
updateThread.looptime_ns = 500000000L; 
// 0.5 s
updateThread.upThreadRdy = NiFpga_True;
status = pthread_create( &thread2,
NULL,
Table_Update_Thread,
&updateThread);
ctable(Table_Title,my_table,8);

updateThread.upThreadRdy = NiFpga_False;
irqThread0.irqThreadRdy = NiFpga_False;
pthread_join(thread, NULL);
pthread_join(thread2, NULL);
status = Irq_UnregisterTimerIrq(
&irqTimer0,
irqThread0.irqContext);

status = MyRio_Close();  
/*Close the myRIO NiFpga Session. */
return status;
}

// Check for speed difference 
//and decide corresponding voltage output
void *Timer_Irq_Thread(void* resource){
ThreadResource* threadResource
 = (ThreadResource*) resource;
double *vref = &((threadResource
->a_table+0)->value);
double *vact = &((threadResource
->a_table+1)->value);
double *VDAout = &((threadResource
->a_table+2)->value);
double *M = &((threadResource
->a_table+3)->value);
double *K = &((threadResource
->a_table+4)->value);
double *B = &((threadResource
->a_table+5)->value);
double *BTI = &((threadResource
->a_table+6)->value);
double *Force = &((threadResource
->a_table+7)->value);
double error;
double VADin;
double alpha;
double spdbuffer[IMAX],posbuffer[IMAX],
forcebuffer[IMAX]; // speed buffer
double *sbp = spdbuffer,*pbp = posbuffer,
*fbp = forcebuffer;// buffer pointer
static double ref[3];
static int i = 1,j = 1;
double BTIp;
int VLimitFlag = 0;
static struct biquad RefSys[] =
{{0, 0, 0,
1, 0, 0, 0, 0, 0, 0, 0}};
/*//---PI controller
int myFilter_ns = 1;  // number of sections
static	struct	biquad myFilter[]={ 	
// define the array 
//of floating point biquads
{3.089502e+01, -2.685314e+01, 
0.000000e+00, 1.000000e+00,
 -1.000000e+00, 0.000000e+00,
  0, 0, 0, 0, 0}
};*/

// in case controller.h not updated
while (threadResource->irqThreadRdy 
== NiFpga_True) {
uint32_t irqAssert = 0;
Irq_Wait( threadResource->irqContext,
TIMERIRQNO,
&irqAssert,
(NiFpga_Bool*) &(threadResource
->irqThreadRdy));

NiFpga_WriteU32( myrio_session,
IRQTIMERWRITE,
*BTI*1000);
NiFpga_WriteBool( myrio_session,
IRQTIMERSETTIME,
NiFpga_True);
if(irqAssert) {
while (l < 20){
wait();
l++;
} // avoid collecting data 
//during initial voltage spike
while (k<=10000) {
IniForce = Aio_Read(&CI0); 
// log a benchmark for 
// force calculation at the beginning
Total = Total +IniForce;
k++;
}
if (i == 1){
IniForce = Total/(k-1);
printf("\nCaliberation Done");
}
// Reading Force from Load Cell
VADin = Aio_Read(&CI0);
if ((VADin-IniForce)<1/124.2 && 
(VADin-IniForce)>-1/124.2){
VADin = IniForce;
}
*Force =(VADin-IniForce)*124.1897;
//Reference System Filter
int RefFilter_ns = 1;
 // No. of sections
alpha = 1/(2*(*M)/(.001*(*BTI))
+*B+(*K)*(*BTI)*.001/2);
RefSys->b0 = alpha;
RefSys->b2 = -alpha;
RefSys->a1 = alpha*((*K)*((*BTI)*.001)
-4*(*M)/(.001*(*BTI)));
RefSys->a2 = alpha*(2*(*M)/
(.001*(*BTI))-*B+(*K)*(.001*(*BTI))/2);
if (*VDAout>VLimit*1000 ||
 *VDAout<-VLimit*1000){ // In mV
VLimitFlag = 1;
}
if (VLimitFlag == 1 && 
*VDAout > (VLimit-1)) {
*Force =0;
}
if (*VDAout <=(VLimit-1)){
VLimitFlag =0;
}
*vref = cascade
(RefFilter_ns,*Force,RefSys);
// Calculate speed, error and voltage
double velocity = vel();
*vact = velocity/2048/5.9/
(*BTI*0.001)*M_PI*0.01834;
error=(*vref-*vact);
//error = (0-*vact);	
// for setting motor to 0 rpm
*VDAout = -cascade
(myFilter_ns ,error, myFilter);
// Cap voltage output
if (*VDAout>VLimit) {
*VDAout = VLimit;
} else if(*VDAout<-VLimit){
*VDAout = -VLimit;
}

if (*VDAout >=VLimit ||*VDAout<=-VLimit){
if (j==1){
putchar_lcd(208);
putchar_lcd(216);
putchar_lcd(220);
}
if (j < 100){
j ++;
}
if (j ==100 ){
j = 1;
}
}
Aio_Write(&CO0, *VDAout);
if (i == 1){
ref[0] = *M;
ref[1] = *K;
ref[2] = *B;
i++;
}
if (ref[0]!=*M || ref[1]!=*K
 || ref[2]!=*B){
sbp = spdbuffer;
pbp = posbuffer;
fbp = forcebuffer;
ref[0] = *M;
ref[1] = *K;
ref[2] = *B;
}
if (sbp < spdbuffer+IMAX) {
*sbp++ = *vact;
*pbp++ = *vref;
*fbp++ = *Force;
}

//round vdaout for display
*vact = roundf(*vact*100)/100;
*vref = roundf(*vref*100)/100;
*Force = roundf(*Force*100)/100;
*VDAout = roundf(*VDAout * 100)/100;
*VDAout = *VDAout*1000;
Irq_Acknowledge(irqAssert);
}
}

MATFILE *mf;
int err;
mf = openmatfile("Lab.mat", &err);
if(!mf) printf
("Can’t open mat file %d\n", err);
matfile_addmatrix
(mf, "Velocity", spdbuffer, IMAX, 1, 0);
matfile_addmatrix
(mf, "V_Reference", posbuffer, IMAX, 1, 0);
matfile_addmatrix
(mf, "Force", forcebuffer, IMAX, 1, 0);
matfile_addmatrix
(mf, "Reference_System", ref, 3, 1, 0);
matfile_addmatrix
(mf, "BTI", &BTIp, 1, 1, 0);
matfile_close(mf);
pthread_exit(NULL);
return NULL;
}

// call update every 0.5s
void *Table_Update_Thread(void* resource){
upThreadResource* threadResource
 = (upThreadResource*) resource;

while (threadResource
->upThreadRdy == NiFpga_True) {
struct timespec timeOut,remains;
timeOut.tv_sec = 0;
timeOut.tv_nsec = 
threadResource->looptime_ns;
nanosleep(&timeOut, &remains);
update();
}
pthread_exit(NULL);
return NULL;
}


double cascade
(int ns, double xin, struct biquad *fa){
static char i;
static double y0;
struct biquad *f;
f = fa;
y0 = xin;            	
 // pass the input to 
 // the first biquad in the cascade
for (i=0; i<ns; i++) {   
 // loop through the "ns" biquads
f->x0 =y0;      	  
 // pass the output to the next biquad
y0 = ( f->b0*f->x0 + f->b1*f->x1 
+ f->b2*f->x2 - f->a1*f->y1 -
 f->a2*f->y2 )/f->a0;
f->x2 = f->x1;  	 
// Update the input history of this biquad
f->x1 = f->x0;
f->y2 = f->y1;  
// Update the output history of this biquad
f->y1 = y0;
f++;      // point to the next biquad
}
return y0; // return the output of the cascade
}

//report counter difference 
// obtianed from encoder counter
double vel(void){
double velocity;
static int count,countn;
static int i = 1;
countn = Encoder_Counter(&encC0);
if (i == 1){
count = countn;
i++;
}
velocity = countn-count;
count = countn;
return (double) velocity;
}

/* create a short wait time*/
void wait(void) {
uint32_t i;
i = 4170000;
while(i>0){
i--;
}
return;
}



\end{verbatim}