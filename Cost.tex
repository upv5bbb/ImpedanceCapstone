\subsection*{Cost and Engineering Economics{{\color{red}\ *}}}
The aerospace industry deals with lots of big heavy parts and machinery. When moving heavy parts such as an airplane turbine, the user operates it much like a joystick controller with buttons that move it either left, right, forward, backward, up or down. The user generally does not have a good idea of the object's movement. If the user was about to feel or obtain resistive feedback on the object's movement, then the object could be controlled much easier. A variable impedance system much like this project can potentially allow a user to move a heavy object just with their own physical force and could feel the resistance and motion of that object. Therefore there is a great demand for such application in the aerospace and automotive manufacturing industry. Such a system could reduce the massive and complicated machinery currently used to move heavy manufactured parts.

Overall, the prototype cost was fairly conservative. The decision to 3D print main design components like the handle and cart, saved the team time and material costs. Parts that were machined, came from extra aluminum in the machine shop that were no extra cost. Time that would have been spent waiting for a lathe or mill and machining parts was used to further refine the control system and brainstorm new design ideas. As a result, management of each team member's time was optimized. Even though two load cells was seen used by Tony in his project, the team chose to use only one load cell to reduce costs. In the end, one load cell turned out to work just as well. The cost of an entire kilogram of PLA filament is used in the cost analysis in Table \ref{cost_analysis}. However even with multiple prints, less than half of the PLA filament was used. The MyRIO and Pittman Motor, the two most costly parts, were supplied by Professor Garbini. So overall the team only spent \$243.12 buying the track setup and necessary hardware.
\begin{table}[H]
	\caption{Cost Analysis}
	\centering
	\label{cost_analysis}
	\begin{tabular}{|l|l|}
		\hline
		\multicolumn{2}{|c|}{Cost Analysis} \\ \hline	\hline
		Load Cell             & \$66.88     \\ \hline
		V-Slot Track Setup    & \$108.64    \\ \hline
		Pittman Motor         & \$318.00    \\ \hline
		MyRIO                 & \$1000.00   \\ \hline
		Switch Box + Relays   & \$35.44     \\ \hline
		PLA Filament          & \$25.00     \\ \hline
		All Fasteners         & \$7.36      \\ \hline	\hline
		Total Project Value   & \$1559.67   \\ \hline
	\end{tabular}
\end{table}